\documentclass[submission,copyright,creativecommons]{eptcs}
\providecommand{\event}{ACL2 2020} % Name of the event you are submitting to
\usepackage{amsmath}
\usepackage{amsfonts}
% \usepackage{breakurl}              % Not needed if you use pdflatex only.
\usepackage{underscore}            % Only needed if you use pdflatex.
\newcommand{\Mod}[1]{\ (\mathrm{mod}\ #1)}
\newcommand{\minus}{\scalebox{0.75}[1.0]{$-$}}
\DeclareMathSymbol{\sneg}{\mathbin}{AMSa}{"39}

%% \setlength{\abovedisplayskip}{1pt}
%% \setlength{\belowdisplayskip}{1pt}
%% \setlength{\abovedisplayshortskip}{1pt}
%% \setlength{\belowdisplayshortskip}{1pt}

%%\setlength{\abovedisplayskip}{3pt}
%%\setlength{\belowdisplayskip}{3pt}

\title{Minimal Fractional Representations of Integers mod $M$ }
\author{David Greve
\email{david@thegreves.com}
}
\def\titlerunning{Minimal Fractions}
\def\authorrunning{D. Greve}

\begin{document}
\maketitle

\begin{abstract}

We say that $x \in \mathbf{Z}/m\mathbf{Z}$ is \emph{represented} by
the fraction $\frac{N}{D}$ when $x*D \equiv N \Mod{M}$ and $D > 0$.
This definition admits many possible fractional representations.  We
say that $\frac{N}{D}$ is a \emph{minimal} representation of $x$ if no
smaller denominator ($D$) results in a numerator with a magnitude less
then $N$.  We introduce a function for computing such fractional
representations and prove that it generates minimal fractions.  We
then prove that every $x \in \mathbf{Z}/m\mathbf{Z}$ has a minimal
fractional representation in which the magnitude of both $N$ and $D$
are less than $\sqrt{M}$.

\end{abstract}

\section{Introduction}

In saying that $x \in \mathbf{Z}/m\mathbf{Z}$ is \emph{represented}
by\footnote{ We might say ``congruent to'' when $D \perp M$ but we don't
require this condition.  }  the fraction $\frac{N}{D}$ when $x*D \equiv N
\Mod{M}$ and $D > 0$ we admit many possible fractional representations
of $x$, some possibly not reduced.  For example, $7 \bmod 17$ has the
following representations:

\[
\{
\,7/1,\, 14/2,\, 4/3,\, 11/4,\, 1/5,\, 8/6,\, 15/7,\, 5/8,\, 12/9 ,\, 2/10,\, 9/11,\, 16/12 ,\,6/13,\, 13/14,\, 3/15,\, 10/16 \,
\}
\]

In this work we employ \emph{signed residues}.  A signed
residue expresses residues closer to 0 as positive values and
residues closer to $M$ as negative values.  
%% The \texttt{smod} function computes signed residues.
%% \footnote{
%% Note that \texttt{(mod (smod x m) m) = (mod x m)}.
%% }
%%
%% \begin{verbatim}
%% (defun smod (x m)
%%   (let ((x (mod x m)))
%%     (if (<= (* 2 x) m) x (- x m))))
%% \end{verbatim}
%%
The signed residues of the numbers $0\dots16 \Mod{17}$ are:

\[
\{
\,0,\,1,\,2,\,3,\,4,\,5,\,6,\,7,\,8,\,\sneg8,\,\sneg7,\,\sneg6,\,\sneg5,\,\sneg4,\,\sneg3,\,\sneg2,\,\sneg1\,
\}
\]

Using signed residues in the numerator, $7 \bmod 17$ has the
following fractional representations:

\[
\{
\,7/1,\,\sneg3/2,\,4/3,\,\sneg6/4,\,1/5,\,8/6,\,\sneg2/7,\,5/8,\,\sneg5/9,\,2/10,\,\sneg8/11,\,\sneg1/12,\,6/13,\,\sneg4/14,\,3/15,\,\sneg7/16\,
\}
\]

We say that $\frac{N}{D}$ is a \emph{minimal} representation of $x$ if
no smaller denominator ($D$) results in a numerator with a magnitude
less then $N$, where the magnitude of $N$ is the absolute value
of the signed residue.
%% \begin{verbatim}
%% (defun mabs (x m)
%%   (abs (smod x m)))
%% \end{verbatim}
In the case of $7 \bmod 17$, ($7/1$) is trivially minimal because we
don't consider denominators less than 1.  ($\sneg3/2$) is also minimal because
the only fraction with a denominator less then 2 is ($7/1$) and 
$\lvert \sneg3 \rvert < \lvert 7\rvert$. 
($\sneg6/4$), however, is not minimal because both ($4/3$)
and ($\sneg3/2$) have numerators and denominators with smaller magnitudes.

Our computation of minimal fractions relies on the following property
of fractional representations, which bears a striking resemblance to
the mediant computation used in the generation of Farey sequences~\cite{Farey}
except that $N_1*D_2 - N_2*D_1$ equals 0 rather than 1 $\Mod{M}$:

\begin{equation*}
x \equiv \frac{N_1}{D_1} \; \land \;
x \equiv \frac{N_2}{D_2} \; \implies \;
x \equiv \frac{N_1 + N_2}{D_1 + D_2}
\end{equation*}

Note that the mediant of two fractions whose numerators differ in sign
produces a residue whose numerator is smaller in magnitude than the
larger of the magnitudes of the two original numerators.  Conceptually
the algorithm takes as input two minimal fractions with differing
signs, initially $x/1$ and $\sneg(M-x)/1$.  It then recursively replaces
the fraction with the largest magnitute numerator with the mediant of
the two minimal fractions until one of the numerators is zero.  The
minimal fractional pairs generated by this algorithm for $7 \bmod 17$
are listed below.

\[
(\sneg10/1,\,7/1),\,(\sneg3/2,\,7/1),\,(\sneg3/2,\,4/3),\,(\sneg3/2,\,1/5),\,(\sneg2/7,\,1/5),\,(\sneg1/12,\,1/5)
\]

Over the pairs of generated fractions we prove that a generalization
of our minimality condition, universally quantified over all non-zero
denominators ($d$), is invariant.  It assert that, if the magnitude of
the residual of $d*x$ is less than the sum of the magnitudes of the
numerators of the pair of fractions, then $d$ must be greater than or
equal to the denominator of the fraction whose numerator has the same
sign as the residual $d*x$ (ie: $d*x/d$ is either not minimal or it is
equal to the denominator of one of the pairs).  Note that under this
generalization the pair of fractions $(\sneg3/2,\,4/3)$ is considered
minimal even though ($4/3$) is not considered minimal under our original
definition because ($\sneg3/2$) has both a numerator and denominator
with smaller magnitudes.

%% If this minimum difference between the numerators is less than the
%% current minimal numerator, then the current fraction is not minimal
%% because the difference between sum or difference between the
%% denominators would result in a fraction with a smaller denominator
%% whose numerator is also smaller.

%% \begin{verbatim}
%% (implies
%%   (and
%%     (< (smod (* a x) m) 0)
%%     (< 0 (smod (* b x) m))
%%     (< (mabs (* a x) m) (mabs (* b x) m)))
%%   (< (mabs (* (+ a b) x) m) (mabs (* b x) m)))
%% \end{verbatim}

In addition to this invariant we show that every $x$ has a fractional
representation in which both $N$ and $D$ are less than $\sqrt{M}$.
%% Informally this assertion is supported by the observation that the
%% size of the $F_n \sim \dfrac{3*n^2}{\pi^2}$.  Including the
%% reciporicals and negations of these fractions, we get approximately
%% $\dfrac{12*n^2}{\pi^2} > n^2$ unique fractions, more than enough to
%% represent the residuals of $M$ if $n = \lfloor\sqrt{M}\rfloor$.  
To prove this we establish that a minimal fraction $N/D$ bounds the
differences between the numerators of every fraction of the same sign
$N_i/D_i \mid D_i < D$ to be greater than $N$.  We then show that,
given $D$ fractions whose numerators are less than $M$, the largest
distance that each numerator can be separated from its closest numeric
neighbor is $M/D$.  Thus, given a minimal fraction pair whose
denominators are less than $\sqrt{M}$ but whose sum is greater than
$\sqrt{M}$, at least one of the two fractions must have a numerator
whose magnitude is less than $\sqrt{M}$.

%% A correlary to this is that the minimal fraction of an integer $x$
%% whose signed residue magnitude is less than $\sqrt{M}$ is $x/1$.  This
%% follows from the observation that the numerator required to make the
%% residue $x*N$ smaller than $x$ will be larger than $\sqrt{M}$.

It is possible for a number to have more than one representation whose
coefficients are less than $\sqrt{M}$.  For example, $12 \bmod 17$ is
represented by both ($\sneg3/4$) and ($2/3$), both of whose
coefficient magnitudes are less than $\sqrt{17}$.  Deciding which is
\emph{the minimum} is a judgement call.  If we say that the
minimum fraction is the one with the smallest maximum coefficient, we
would select ($2/3$) as the minimum fractional representation of $12 \bmod 17$.  
Using this minimality criteria the minimum fractional
representations for each of the numbers $1\dots16 \Mod{17}$ are:

\[
\{
\,1,\,2,\,3,\,4,\,\sneg2/3,\,1/3,\,\sneg3/2,\,\sneg1/2,\,1/2,\,3/2,\,\sneg1/3,\,2/3,\,4,\,\sneg3,\,\sneg2,\,\sneg1\,
\}
\]

\section{Conclusion}

In the cryptographic community there is interest in finding smooth
numbers that result from specific computations.  The quadratic sieve
algorithm~\cite{Sieve}, for example, attempts to find small numbers (numbers on the
order of $\sqrt{M}$) in hopes of factoring them into a smooth factor
base.  We have shown that any residue relatively prime to $M$ can be
expressed as a quotient of two numbers less than $\sqrt{M}$.

\bibliography{generic}{}
\bibliographystyle{eptcs}
\end{document}
