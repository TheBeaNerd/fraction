\documentclass[submission,copyright,creativecommons]{eptcs}
\providecommand{\event}{ACL2 2020} % Name of the event you are submitting to
\usepackage{setspace}
\usepackage{amsmath}
\usepackage{amsfonts}
% \usepackage{breakurl}              % Not needed if you use pdflatex only.
\usepackage{underscore}            % Only needed if you use pdflatex.
\newcommand{\Mod}[1]{\ (\mathrm{mod}\ #1)}
\newcommand{\minus}{\scalebox{0.75}[1.0]{$-$}}
\DeclareMathSymbol{\sneg}{\mathbin}{AMSa}{"39}

%% \setlength{\abovedisplayskip}{1pt}
%% \setlength{\belowdisplayskip}{1pt}
%% \setlength{\abovedisplayshortskip}{1pt}
%% \setlength{\belowdisplayshortskip}{1pt}

%%\setlength{\abovedisplayskip}{3pt}
%%\setlength{\belowdisplayskip}{3pt}

\title{Minimal Fractional Representations of Integers mod $M$ }
\author{David Greve
\email{david@thegreves.com}
}
\def\titlerunning{Minimal Fractions}
\def\authorrunning{D. Greve}

\begin{document}
\maketitle

\begin{abstract}

We say that the fraction $\frac{N}{D}$ \emph{represents} $x \in
\mathbf{Z}/m\mathbf{Z}$ when $x*D \equiv N \Mod{M}$.  This definition
admits many possible fractional representations.  We say that
$\frac{N}{D}$ is a \emph{minimal} representation of $x$ if no smaller
denominator ($D$) results in a numerator with a magnitude less then
$N$.  We introduce a function for computing such fractional
representations and prove that it generates minimal fractions.  We
also prove that every $x \in \mathbf{Z}/m\mathbf{Z}$ has a minimal
fractional representation in which the magnitude of $N$ and $D$
are less or equal to $\sqrt{M}$.

\end{abstract}

\section{Introduction}

In saying that the fraction $\frac{N}{D}$ \emph{represents}\footnote{
  We might say ``congruent to'' when $D \perp M$ but we don't require
  this condition.  }  $x \in \mathbf{Z}/m\mathbf{Z}$ when $x*D \equiv
N \Mod{M}$ we admit many possible fractional representations of $x$,
some possibly not reduced.  For example, $7 \Mod{17}$ has the
following representations:
\begin{spacing}{1.0}
{\small
\[
\{
\,7/1,\, 14/2,\, 4/3,\, 11/4,\, 1/5,\, 8/6,\, 15/7,\, 5/8,\, 12/9 ,\, 2/10,\, 9/11,\, 16/12 ,\,6/13,\, 13/14,\, 3/15,\, 10/16,\, 0/17 \,
\}
\]
}
\end{spacing}
\begin{spacing}{1.0}
In this work we consider both positive and negative residues in the
numerator.  A positive residue is defined as $(x*D \bmod M)$ for $0 <
D \leq Q$ while a negative residue is defined as $(x*D \bmod M) - M$
for $0 \leq D < Q$.  Using the negative residue the fractional
representations of $7 \Mod{17}$ are:
\end{spacing}
\begin{spacing}{1.0}
{\footnotesize
\[
\{
\sneg17/0,\,
\sneg10/1,\,
\sneg3/2,\,
\sneg13/3,\,
\sneg6/4,\,
\sneg16/5,\,
\sneg9/6,\,
\sneg2/7,\,
\sneg12/8,\,
\sneg5/9,\,
\sneg15/10,\,
\sneg8/11,\,
\sneg1/12,\,
\sneg11/13,\,
\sneg4/14,\,
\sneg14/15,\,
\sneg7/16\,
\}
\]
}
\end{spacing}
Within the same residue class (positive or negative), we say that
$\frac{N}{D}$ is a \emph{minimal} representation of $x$ if no
denominator smaller than ($D$) results in a numerator with a magnitude
less then $N$.  ($7/1$) is minimal for $7 \Mod{17}$ simply because we
don't consider positive residues with denominators less than 1.  The
fraction ($\sneg3/2$) is also minimal because $\lvert \sneg3 \rvert$ is
less then the magnitude of the numerators of both negative fractions
with denominators less than $2$, ($-17/0$) and ($-10/1$).
($\sneg6/4$), however, is not minimal because ($\sneg3/2$) has both a
smaller magnitude numerator and a smaller denominator.

Our proof of correctness actually requires a stronger, more general
notion of minimality that is invariant over our algorithm for computing
minimal fractions.  This more general invariant is expressed with
respect to a pair of fractions: one negative residual and one positive
residual.  The pair is considered minimal if, for all possible
denominators ($d$), if the magnitude of either the positive or
negative residual of $d*x$ is less than the \emph{sum} of the
magnitudes of the numerators of the pair of fractions, then $d$ must
be greater than or equal to the denominator of the fraction with the
same residual sign.  Under this generalization the pair of fractions
$(\sneg3/2,\,4/3)$ is considered minimal because no denominator less
than $3$ has a positive residual and no denominator less than $2$ has
a negative residual less than $\lvert \sneg3 \rvert + \lvert 4 \rvert
= 7$.

Our computation of minimal fractions relies on the following property
of the mediant computation\footnote{As used in the generation of Farey
  sequences~\cite{Farey} except that, in our case, $N_1*D_2 - N_2*D_1$
  is equal to $M$ rather than $1$}:

\begin{equation*}
x \equiv \frac{N_1}{D_1} \; \land \;
x \equiv \frac{N_2}{D_2} \; \implies \;
x \equiv \frac{N_1 + N_2}{D_1 + D_2} \; \; \Mod{M}
\end{equation*}

Our algorithm takes as input two minimal fractions with differing
signs, initially $(x-M)/0$ and $x/1$ (which are trivially minimal).  It
then recursively replaces the fraction with the largest magnitude
numerator with the mediant of the two original fractions until one of
the numerators is zero.  The key to the termination of our algorithm
is the observation that the mediant of two fractions whose numerators
differ in sign is a fraction whose numerator is smaller in magnitude
than the larger of the magnitudes of the two original numerators.  We
prove that this algorithm preserves our notion of minimal fractional pairs.
The minimal fractional pairs generated by our algorithm
for $7 \Mod{17}$ are listed below.

{\small
\[
(\sneg17/0,\,7/1),\,
(\sneg10/1,\,7/1),\,(\sneg3/2,\,7/1),\,(\sneg3/2,\,4/3),\,(\sneg3/2,\,1/5),\,(\sneg2/7,\,1/5),\,(\sneg1/12,\,1/5),\,
(\sneg1/12,\,0/17)\,
\]
}

In addition to our minimality invariant we prove that every $x$ has a fractional
representation in which both $N$ and $D$ are less than or equal to
$\sqrt{M}$.  This result follows from the fact that $N_1*D_2 - N_2*D_1 = M$
is invariant.  Informally, consider the case where $N_1 < \sqrt{M}$ and
$N_2 < \sqrt{M}$ but $N_1 + N_2 \geq \sqrt{M}$.  If both $D_1 > \sqrt{M}$ and
$D_2 > \sqrt{M}$, then in the following step, $(N_1 + N_2)*D_i$ will be
greater than $M$, violating our invariant.  Thus, at
least one $D_i < \sqrt{M}$.

It is possible for a number to have more than one representation whose
coefficients are less than $\sqrt{M}$.  For example, $12 \Mod{17}$ is
represented by both ($\sneg3/4$) and ($2/3$), both of whose
coefficient magnitudes are less than $\sqrt{17}$.  Deciding which is
\emph{the minimum} is a judgment call.  We say that the minimum
fraction is the one with the smallest maximum coefficient, resolving
ties with the smaller denominator.  Under those conditions the
minimum fractional representation of $12 \Mod{17}$ is ($2/3$).  Using
this minimality criteria the minimum fractional representations for
each of the numbers $1\dots16 \Mod{17}$ are:

\[
\{
\,1,\,2,\,3,\,4,\,\sneg2/3,\,1/3,\,\sneg3/2,\,\sneg1/2,\,1/2,\,3/2,\,\sneg1/3,\,2/3,\,\sneg4,\,\sneg3,\,\sneg2,\,\sneg1\,
\}
\]

\section{Conclusion}

We have verified an algorithm for computing minimal fractional
representations $\frac{N}{D}$ of numbers $x \in
\mathbf{Z}/m\mathbf{Z}$.  We also proved that all such numbers have a
minimal representation in which the magnitude of both $N$ and $D$ are
less than or equal to $\sqrt{M}$.  In the cryptographic community
there is interest in finding smooth numbers that result from specific
computations.  The quadratic sieve algorithm~\cite{Sieve}, for
example, attempts to find small numbers (numbers on the order of
$\sqrt{M}$) in hopes of factoring them into a smooth factor base.  We
show that any residue relatively prime to $M$ can be expressed as a
quotient of two numbers less than or equal to $\sqrt{M}$.

\bibliography{generic}{}
\bibliographystyle{eptcs}
\end{document}
